\documentclass{codLab}
\usepackage{listings}
\begin{document}
	\firsttitlepage{姓名}{学号}{学院}{专业}{任课教师}{报告日期}
	\secondtitlepage{实验项目名称}{学生姓名}{学号}{紫金港东四509室}{2024}{12}{13}
	\section{操作方法与实验步骤}
	本节重点介绍实验的具体过程,包括:代码设计层次结构图及说明、源代码(包括注释)、PC机上进行的关键步骤截图及说明、调试过程等,这部分的内容应当与实际操作过程和结果相符。本节也可以再细分小节,要求同上。
	
	代码示例:
	
\begin{lstlisting}[caption=xdc代码,language=xdc]
	# LE即总开关
	set_property PACKAGE_PIN AA10 [get_ports LE]  
	set_property IOSTANDARD LVCMOS15 [get_ports LE]  
	
	# buzzer_on是蜂鸣器  
	set_property PACKAGE_PIN AF25 [get_ports buzzer_on]  
	set_property IOSTANDARD LVCMOS33 [get_ports buzzer_on]
	
	# clk_origin是频率为100MHz的板载时钟  
	set_property PACKAGE_PIN AC18 [get_ports clk_origin]  
	set_property IOSTANDARD LVCMOS18 [get_ports clk_origin]  
	create_clock -period 10.000 -name clk [get_ports "clk_origin"]
	
\end{lstlisting}
\begin{lstlisting}[title=verilog代码,language=verilog]
	module Beat_Show(
		input wire clk_origin,
		//系统时钟信号
		input wire clk,
		//经过速度选择后选择的时钟频率
		input wire LE_PAUSE,
		//总控使能信号,控制整个模块的启动与否
		output wire LED_CLK,
		output wire LED_CLR,
		output wire LED_DO,
		output wire LED_EN,
		//将LED驱动的结果输出给实验板上的LED板块,控制LED灯的状态
		output wire[1:0]led,
		output wire LED2
		//控制指示灯的亮灭,指示此刻是否按下按钮
	);
\end{lstlisting}
	\section{实验结果与分析}
	1.这里应给出详实的实验结果。分析应有条理,要求采用规范的书面语。
	2.实验四后每个实验都需要做模拟,要到每一个模拟结果的每一段结果做分析说明。
	3.对下载到NEXYS实验台的图片结果做分析说明。
	4.原则上要求使用图片与文字结合的形式说明,因为word和PDF文档不支持视频,所以请不要使用视频文件。
	5.图片请在垂直方向,不要横向。不要用很大的图片,请先做裁剪操作。
	图片命令示例:
	\begin{singlefigure}[标题]{浙江大学}
		这是一些居中排版的对图片的注释。
	\end{singlefigure}
	图片命令拥有第二个可选参数,用于控制图片的长度,默认为0.7总长。
	\begin{singlefigure}[控制为0.5总长]{浙江大学}[0.5]
		第二个可选参数填0.5.
	\end{singlefigure}
	\section{讨论、心得}
	简要地叙述一下实验过程中的感受,以及其他的问题描述和自己的感想。特别是实验中遇到的困难,最后如何解决的。在用verilog代码写程序时遇到语法或其他错误,如何修改解决的。

\end{document}